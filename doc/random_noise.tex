\documentclass[a4paper,draft,12pt]{report}

\usepackage{amsmath}
\usepackage{amssymb}

\allowdisplaybreaks

\title{Broadcast algorithms in SINR model with random noise}


\begin{document}

\maketitle

\chapter{Introduction}

\chapter{Model}

\section{Signal-to-interference ratio}

\begin{equation}
\label{eq:sinr}
\frac{\frac{P}{d^\alpha}}{I + N}
\end{equation}

\section{Random noise}

In the real world the noise value is not constant. It depends on many physical factors like weather or even cosmic noise. Using random values approximating real noise seems to be a good idea.

Random noise can be modelled quite as a Generalized Extreme Value  \cite{gev} distribution with some parameters $\mu, \sigma, \xi$. We assume $\xi \neq 0$. Cumulative distribution function for GEV($\mu, \sigma, \xi$) distribution is:

\begin{equation}
\label{eq:gev_cdf}
\textbf{P}(N \leq x) = e^{-(1 + \xi \frac{x - \mu}{\sigma}) ^ {-\frac{1}{\xi}}}
\end{equation}

The values of $\xi, \mu$ and $\sigma$ can be found experimentally.

\section{Range}

We define transmission range $r_i$ of a node $i$ as a maximum distance at which the node can succesfully transmit its signal when there is no interference other than background noise. This is easy to obtain from \eqref{eq:sinr}:

\begin{align*}
\beta &= \frac{\frac{P_i}{r_i^\alpha}}{0 + N} \\
\beta N &= \frac{P_i}{r_i^\alpha} \\
r_i^\alpha &= \frac{P_i}{\beta N} \\
r_i &= \left( \frac{P_i}{\beta N} \right)^{\frac{1}{\alpha}}
\end{align*}

Considering random noise, transmission range of a node can vary due to current noise. We define probable transmission range $\tilde{r}_i$ as a distance at which we have 0.95 probability of a successful transmission.  We have:

\begin{align*}
\textbf{P}  \left( \beta \leq \frac{\frac{P_i}{\tilde{r}_i^\alpha}}{0 + N} \right) &= 0.95 \\
\textbf{P}  \left( N \leq \frac{P_i}{\beta \tilde{r}_i^\alpha} \right) &=  0.95
\end{align*}

and applying \eqref{eq:gev_cdf} we have:

\begin{align}
e^{-\left(1 + \xi \frac{\frac{P_i}{\beta \tilde{r}_i^\alpha}  - \mu}{\sigma}\right) ^ {-\frac{1}{\xi}}}& = 0.95 \nonumber \\
\left(1 + \xi \frac{\frac{P_i}{\beta \tilde{r}_i^\alpha}  - \mu}{\sigma}\right) ^ {-\frac{1}{\xi}} &= (- \ln 0.95) \nonumber \\
\xi \frac{\frac{P_i}{\beta \tilde{r}_i^\alpha}  - \mu}{\sigma} &= (- \ln 0.95)^{-\xi} - 1\nonumber \\
\frac{P_i}{\beta \tilde{r}_i^\alpha}  - \mu &= \frac{\sigma (- \ln 0.95)^{-\xi} - 1}{\xi} \nonumber\\
\frac{1}{\tilde{r}_i^\alpha} &= \frac{\beta}{P_i} \left( \frac{\sigma (- \ln 0.95)^{-\xi} - 1}{\xi} + \mu \right)\nonumber \\
\tilde{r}_i^\alpha  &= \left[ \frac{\beta}{P_i} \left( \frac{\sigma (- \ln 0.95)^{-\xi} - 1}{\xi} + \mu \right) \right]^{-1}\nonumber \\
\tilde{r}_i  &= \left[ \frac{\beta}{P_i} \left( \frac{\sigma (- \ln 0.95)^{-\xi} - 1}{\xi} + \mu \right) \right]^{-\frac{1}{\alpha}}
\end{align}

\chapter{Network graph}

Network graph is a undirected graph $G = (V, E)$, where $V = \{v_1, \ldots, v_n\}$ is a set of nodes (points) and $E \in V^2$ is a set of links between nodes. For each node $v_i$ we define $E(v_i)$ to be a set of neighbors of $v_i$, ie. $E(v_i) = \{ v_j \in V : (v_i, v_j) \in E\}$.

We used three methods for generating network graphs:
\begin{enumerate}
\item uniform networks,
\item social networks,
\item gadget networks
\end{enumerate}

\noindent Each generated network is strongly connected.

\section{Uniform networks}

Uniforms networks are random graphs generated using a uniform distribution. Each node has a position in $[0, S] \times [0, S]$ square.

\section{Social networks}

Social networks tries to model human behaviour of connecting into larger group. Nodes are generated within a square $[0, S] \times [0, S]$. The square is divided into subsquares of size $\epsilon$. We define $s_{i,j}$ to be a set of nodes in a subsquare in i-th row and j-th column. To a subsquare $s_{i,j}$ we assign a weight:

\begin{equation*}
w_{i,j} = \left| \bigcup_{v \in s_{i,j}} E(E(v)) \right|
\end{equation*}

With probability $1 - \gamma$ we choose a subsquare $s$ according to the weights, and put new node with random position (using uniform distribution) within the chosen subsquare. With probablity $\gamma$ we use uniform network algorithm to generate a new node.

\section{Gadget networks}

Gadget network is a special class of network designed to be  ``hard" for algorithms.

\chapter{Algorithms}

We present few broadcast algorithms for SINR model.

\section{Density-aware}

We divide the surface into squares of size $\epsilon$. Each node knows how the number $\Delta$ of all nodes in its square. After end of a round, each nodes that already got the message decides to transmit with probability:

\begin{equation}
\label{eq:dens_alg_prob}
P = C  \epsilon^3  \min\{4, \frac{1}{\alpha - 2}, \ln n\}
\end{equation}

where $C$ is a constant, $n$ is the network size and $\alpha$ is a model  parameter.

\chapter{Experiments and results}

\begin{thebibliography}{9}

\bibitem{sinrsurvey}
Olga Goussevskaia , Yvonne-Anne
Pignolet, Roger Wattenhofer,
\emph{Efficiency of Wireless Networks:
Approximation Algorithms
for the Physical Interference Model},
2010

\bibitem{gev}
Xu Su, Rajendra V. Boppana,
\emph{On the impact of noise on mobile ad hoc networks},
2007, ACM

\end{thebibliography}

\end{document}