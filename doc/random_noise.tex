\documentclass[a4paper,draft,12pt]{report}

\usepackage{amsmath}
\usepackage{amssymb}
\usepackage{algorithm}
\usepackage{algpseudocode}

\allowdisplaybreaks

\title{Broadcast algorithms in SINR model with random noise}


\begin{document}

\maketitle

\chapter{Introduction}

\chapter{Model}

\section{Signal-to-interference ratio}

\begin{equation}
\label{eq:sinr}
\frac{\frac{P}{d^\alpha}}{I + N}
\end{equation}

\section{Random noise}

In the real world the noise value is not constant. It depends on many physical factors like weather or even cosmic noise. Using random values approximating real noise seems to be a good idea.

Random noise can be modelled quite as a Generalized Extreme Value  \cite{gev} distribution with some parameters $\mu, \sigma, \xi$. We assume $\xi \neq 0$. Cumulative distribution function for GEV($\mu, \sigma, \xi$) distribution is:

\begin{equation}
\label{eq:gev_cdf}
\textbf{P}(N \leq x) = e^{-(1 + \xi \frac{x - \mu}{\sigma}) ^ {-\frac{1}{\xi}}}
\end{equation}

The values of $\xi, \mu$ and $\sigma$ can be found experimentally.

\section{Range}

We define transmission range $r_i$ of a node $i$ as a maximum distance at which the node can succesfully transmit its signal when there is no interference other than background noise. This is easy to obtain from \eqref{eq:sinr}:

\begin{align*}
\beta &= \frac{\frac{P_i}{r_i^\alpha}}{0 + N} \\
\beta N &= \frac{P_i}{r_i^\alpha} \\
r_i^\alpha &= \frac{P_i}{\beta N} \\
r_i &= \left( \frac{P_i}{\beta N} \right)^{\frac{1}{\alpha}}
\end{align*}

Considering random noise, transmission range of a node can vary due to current noise. We define probable transmission range $\tilde{r}_i$ as a distance at which we have 0.95 probability of a successful transmission.  We have:

\begin{align*}
\textbf{P}  \left( \beta \leq \frac{\frac{P_i}{\tilde{r}_i^\alpha}}{0 + N} \right) &= 0.95 \\
\textbf{P}  \left( N \leq \frac{P_i}{\beta \tilde{r}_i^\alpha} \right) &=  0.95
\end{align*}

and applying \eqref{eq:gev_cdf} we have:

\begin{align}
e^{-\left(1 + \xi \frac{\frac{P_i}{\beta \tilde{r}_i^\alpha}  - \mu}{\sigma}\right) ^ {-\frac{1}{\xi}}}& = 0.95 \nonumber \\
\left(1 + \xi \frac{\frac{P_i}{\beta \tilde{r}_i^\alpha}  - \mu}{\sigma}\right) ^ {-\frac{1}{\xi}} &= (- \ln 0.95) \nonumber \\
\xi \frac{\frac{P_i}{\beta \tilde{r}_i^\alpha}  - \mu}{\sigma} &= (- \ln 0.95)^{-\xi} - 1\nonumber \\
\frac{P_i}{\beta \tilde{r}_i^\alpha}  - \mu &= \frac{\sigma (- \ln 0.95)^{-\xi} - 1}{\xi} \nonumber\\
\frac{1}{\tilde{r}_i^\alpha} &= \frac{\beta}{P_i} \left( \frac{\sigma (- \ln 0.95)^{-\xi} - 1}{\xi} + \mu \right)\nonumber \\
\tilde{r}_i^\alpha  &= \left[ \frac{\beta}{P_i} \left( \frac{\sigma (- \ln 0.95)^{-\xi} - 1}{\xi} + \mu \right) \right]^{-1}\nonumber \\
\tilde{r}_i  &= \left[ \frac{\beta}{P_i} \left( \frac{\sigma (- \ln 0.95)^{-\xi} - 1}{\xi} + \mu \right) \right]^{-\frac{1}{\alpha}}
\end{align}

\chapter{Network graph}

Network graph is a undirected graph $G = (V, E)$, where $V = \{v_1, \ldots, v_n\}$ is a set of nodes (points) and $E \in V^2$ is a set of links between nodes. For each node $v_i$ we define $E(v_i)$ to be a set of neighbors of $v_i$, ie. $E(v_i) = \{ v_j \in V : (v_i, v_j) \in E\}$.

We used three methods for generating network graphs:
\begin{enumerate}
\item uniform networks,
\item social networks,
\item gadget networks
\end{enumerate}

\noindent Each generated network is strongly connected.

\section{Uniform networks}

Uniforms networks are random graphs generated using a uniform distribution. Each node has a position in $[0, S] \times [0, S]$ square.

\begin{algorithm}
\caption{Uniform network generation}
\label{a:uniform}
\begin{algorithmic}
\Function{uniform-node}{$S$}
	\State $x \gets \text{random}(0, S)$
	\State $y \gets \text{random}(0, S)$
	\State \Return $(x, y)$
\EndFunction
\end{algorithmic}
\end{algorithm}

\section{Social networks}

Social networks tries to model human behaviour of connecting into larger group. Nodes are generated within a square $[0, S] \times [0, S]$. The square is divided into tiles of size $\epsilon$. We define $t_{i,j}$ to be a set of nodes in a tile in i-th row and j-th column. To a tile $t_{i,j}$ we assign a weight:

\begin{equation*}
w_{i,j} = \left|seclinks_{t_{i,j}} \right|
\end{equation*}
\begin{equation*}
seclinks_t =  \bigcup_{v \in t} E(E(v))
\end{equation*}

With probability $1 - \gamma$ we choose a tile $t$ according to the weights, and put new node with random position (using uniform distribution) within the chosen tile. With probablity $\gamma$ we use uniform network algorithm to generate a new node.

The algorithm keeps track of current $seclinks$ values and updates them in \textsc{update-weights} after every new node.

\begin{algorithm}
\caption{Social network generation}
\label{a:social}
\begin{algorithmic}
\Function{update-weights}{v}
	\State let $v \in t$
	\For{$v' \in E(v)$}
		\State let $v' \in t'$
		\State $seclinks_{t'} \gets seclinks_{t'} \cup (E(v)  \setminus \{v'\})$
		\For{$v'' \in E(v')$}
			\If{$v \neq v'$'}
				\State let $v'' \in t''$
				\State $seclinks_{t''} \gets seclinks_{t''} \cup \{v\}$
				\State $seclinks_t \gets seclikns_t \cup \{v''\}$
			\EndIf
		\EndFor
	\EndFor
\EndFunction

\Statex

\Function{social-node}{$S, e, \gamma$}
	\If{$\text{random()} < \gamma$}
		\State \Return \textsc{uniform-node}(S)
	\Else
		\State $tile \gets \textsc{weight-random}(W)$
		\State $(e_x, e_y) \gets \textsc{uniform-node}(e)$
		
		\State \Return $(tile.x + e_x, tile.y + e_y)$
	\EndIf
\EndFunction
\end{algorithmic}
\end{algorithm}

\section{Gadget networks}

Gadget network is a special class of network designed to be  ``hard" for algorithms.

\chapter{Algorithms}

We present few broadcast algorithms for SINR model. Each algorithm is run in distributed network and consist of three function---\textsc{Init}, \textsc{on-round-end} and \textsc{is-done}---executed for every node in the network.

\textsc{init} is executed globally at the beginning. Its purpose is to setup and initialize each node. The return value indicated the set of nodes that fiest hold the message.

\textsc{on-round-end} is executed for every node at the end of each round. The node knows what is the current round number and whether it received any message. If the return  value  is true, then the node will be transmiting in the round. Otherwise it will be receiving.

\textsc{is-done} is executed globally at the end of each round. The return value is true, when the algorithm is done.

\section{Density-aware algorithm}

We divide the surface into tiles of size $\epsilon$. Each node knows how the number $\Delta$ of all nodes in its tile. After end of a round, each node $v$ that already got the message decides to transmit with probability :

\begin{align}
\label{eq:dens_alg_prob}
P_v &= C  \frac{d}{\Delta} \\
d &= \epsilon^3  \min\{4, \frac{1}{\alpha - 2}, \ln n\} \nonumber
\end{align}

where $C$ is a constant, $n$ is the network size and $\alpha$ is a model  parameter.

\begin{algorithm}
\caption{Density-aware algorithm}
\label{a:density}
\begin{algorithmic}
\Function{init}{V}
        \State $d \gets C e^3 \min\{4, \frac{1}{\alpha - 2}, \log N\}$
        \State $\forall v_i \in V \, active_i \gets 0$
        \State $\forall v_i \in V \,density_i \gets |\{v \in V :  v \text{ is in the same tile as } v_i\}|$
        \State $\forall v_i \in V \,P_i \gets \frac{d}{density_i}$
        \State \Return $\{\text{random}(V)\}$
\EndFunction

\Statex

\Function{is-done}{}
	\If{$\forall v_i \in V \, active_i = true$}
		\State \Return true
	\Else
		\State \Return false
	\EndIf
\EndFunction

\Statex

\Function{on-round-end}{$S, v_i, messages, round_number$}
	\If{$messages$}
		\State $active_i \gets 1$
	\EndIf
	\If{$active_i$}
		\State \Return $\text{random()} < P_i$
	\Else
		\State \Return 0
	\EndIf 
\EndFunction
\end{algorithmic}
\end{algorithm}

\chapter{Experiments and results}

\begin{thebibliography}{9}

\bibitem{sinrsurvey}
Olga Goussevskaia , Yvonne-Anne
Pignolet, Roger Wattenhofer,
\emph{Efficiency of Wireless Networks:
Approximation Algorithms
for the Physical Interference Model},
2010

\bibitem{gev}
Xu Su, Rajendra V. Boppana,
\emph{On the impact of noise on mobile ad hoc networks},
2007, ACM

\end{thebibliography}

\end{document}